% !TEX encoding = UTF-8
% !TEX TS-program = pdflatex
% !TEX root = ../tesi.tex



%**************************************************************
% Acronimi
%**************************************************************
\newacronym[description={\glslink{API}{Application Program Interface}}]
{API}{API}{Application Program Interface}

\newacronym[description={\glslink{uml}{Unified Modeling Language}}]
{uml}{UML}{Unified Modeling Language}

\newacronym[description={\glslink{IDE}{Integrated Development Enviroment}}]
{IDE}{IDE}{Integrated Development Enviroment}

\newacronym[description={\glslink{UI}{User Interface}}]
{UI}{UI}{User Interface}

\newacronym[description={\glslink{MVC}{Model View Controller}}]
{MVC}{MVC}{Model View Controller}

\newacronym[description={\glslink{REST}{REpresentational State Transfer}}]
{REST}{REST}{REpresentational State Transfer}

\newacronym[description={\glslink{JSON}{JavaScript Object Notation}}]
{JSON}{JSON}{JavaScript Object Notation}

\newacronym[description={\glslink{MVVM}{Model-View-ViewModel}}]
{MVVM}{MVVM}{Model-View-ViewModel}

\newacronym[description={\glslink{URL}{Uniform Resource Locator}}]
{URL}{URL}{Uniform Resource Locator}

\newacronym[description={\glslink{HTML}{HyperText Markup Language}}]
{HTML}{HTML}{HyperText Markup Language}

\newacronym[description={\glslink{DOM}{Document Object Model}}]
{DOM}{DOM}{Document Object Model}


%**************************************************************
% Glossario
%**************************************************************
\newglossaryentry{start-up}
{
  name=\glslink{start-up}{start-up},
  text=start-up,
  sort=start-up,
  description={è un'azienda che è nella fase iniziale di avvio delle attività}
}
\newglossaryentry{metadati}
{
  name=\glslink{metadati}{metadati},
  text=metadati,
  sort=metadati,
  description={sono le informazioni che permettono la costruzione dei dati}
}
\newglossaryentry{byte}
{
  name=\glslink{byte}{byte},
  text=byte,
  sort=byte,
  description={è generalmente una sequenza di 8 bit. Con i multipli del byte si identifica l'unità di misura della capacità di memoria}
}
\newglossaryentry{frontend}
{
  name=\glslink{frontend}{frontend},
  text=frontend,
  sort=frontend,
  description={nell'architettura di un applicazione si intende la parte di software visibile all'utente}
}
\newglossaryentry{backend}
{
  name=\glslink{backend}{backend},
  text=backend,
  sort=backend,
  description={nell'architettura di un applicazione si intende la parte di software con la quale l'utente non interagisce direttamente, che generalmente gestisce l'archiviazione dei dati e la  logica di busness}
}
\newglossaryentry{object storage}
{
  name=\glslink{object storage}{object storage},
  text=object storage,
  sort=object storage,
  description={è un architettura per la memorizzazione dei dati, dove questi vengono gestiti come oggetti}
}
\newglossaryentry{libreria}
{
  name=\glslink{libreria}{libreria},
  text=libreria,
  sort=libreria,
  description={è un insieme di funzioni e strutture dati non direttamente eseguibili utilizzati a supporto di un software}
}
\newglossaryentry{form}
{
  name=\glslink{form}{form},
  text=form,
  sort=form,
  description={è la parte di interfaccia utente di un applicazione che consente all'utente l'iserimento di dati}
}
\newglossaryentry{task}
{
  name=\glslink{task}{task},
  text=task,
  sort=task,
  description={in un sistema di organizzazione del lavoro è una specifica attività che fa parte di un progetto}
}
\newglossaryentry{repository}
{
  name=\glslink{repository}{repository},
  text=repository,
  sort=repository,
  description={è un sistema di archiviazione di file e dati utilizzato per gestire il codice di un progetto}
}
\newglossaryentry{workflow}
{
  name=\glslink{workflow}{workflow},
  text=workflow,
  sort=workflow,
  description={nella gestione dei progetti è il flusso di lavoro che un attività deve seguire dalla sua creazione al suo completamento}
}
\newglossaryentry{plugin}
{
  name=\glslink{plugin}{plugin},
  text=plugin,
  sort=plugin,
  description={è un programma a supporto di un altro che ne estende le funzionalità}
}
\newglossaryentry{framework}
{
  name=\glslink{framework}{framework},
  text=framework,
  sort=framework,
  description={è l'architettura logica sulla quale il software viene progettato e realizzato, utilizzato per facilitare lo sviluppo e il mantenimento del codice}
}
\newglossaryentry{jobs}
{
  name=\glslink{jobs}{jobs},
  text=jobs,
  sort=jobs,
  description={nella nostra applicazione con questo termine si intende i procedimenti per i quali è già stato fatto l'upload verso il backend}
}
\newglossaryentry{open-source}
{
  name=\glslink{open-source}{open-source},
  text=open-source,
  sort=open-source,
  description={indica una particolare la licenza di un software, che consente la libera distribuzione la modifica e lo studio}
}
\newglossaryentry{routing}
{
  name=\glslink{routing}{routing},
  text=routing,
  sort=routing,
  description={indica la parte di gestione dell'indirizzamento delle URL e dei percorsi (path) in un applicazione web}
}
\newglossaryentry{endpoint}
{
  name=\glslink{endpoint}{endpoint},
  text=endpoint,
  sort=endpoint,
  description={nella comunicazione frontend-backend tramite API è il punto che il server espone per ricevere richieste ed inviare risposte}
}
\newglossaryentry{sessione}
{
  name=\glslink{sessione}{sessione},
  text=sessione,
  sort=sessione,
  description={nelle applicazioni web, indica l'attività svolta dall'utente da quando si collega all'applicazione tramite browser a quando il collegamento viene chiuso}
}
\newglossaryentry{stato}
{
  name=\glslink{stato}{stato},
  text=stato dell'applicazione,
  sort=stato,
  description={è l'insieme di input di dati e azioni che determinano l'output in un dato momento.}
}
\newglossaryentry{brano}
{
  name=\glslink{brano}{brano},
  text=brano,
  sort=brano,
  description={nelle tracce audio del nostro sistema di registrazione, è la traccia più esterna che può contenere vari sottobrani. Possono essercene più di uno per CD.}
}
\newglossaryentry{sottobrano}
{
  name=\glslink{sottobrano}{sottobrano},
  text=sottobrano,
  sort=sottobrano,
  description={nelle tracce audio del nostro sistema di registrazione, è la traccia più interna che è figlia di un brano. Possono essercene più di una per ogni brano.}
}
\newglossaryentry{canali}
{
  name=\glslink{canali}{canali},
  text=canale audio,
  sort=canale,
  description={nell'aula di tribunale, corrisponde alla registrazione di uno dei microfoni disponibili in aula nei quali uno dei presenti parla. Ad ogni canale audio corrisponde
      una traccia.}
}
\newglossaryentry{traccia mixer}
{
  name=\glslink{traccia mixer}{traccia mixer},
  text=traccia mixer,
  sort=traccia mixer,
  description={nel nostro sistema di registrazione, è la traccia risultato del mixaggio delle tracce corrispondendi ai singoli canali. In particolare ci sono 7 microfoni
      in aula, ognuno viene registrato su una traccia audio, la traccia mixer è quella che registra tutto l'insieme.}
}
\newglossaryentry{parsing}
{
  name=\glslink{parsing}{parsing},
  text=parsing,
  sort=parsing,
  description={significa letteralmente analisi, nel nostro contesto si riferisce all'analisi e all'interpretazione che la libreria creata fa sui metadati che si trovano nel file cronologia}
}
\newglossaryentry{oggetti javascript}
{
  name=\glslink{oggetti javascript}{oggetti javascript},
  text=oggetto javascript,
  sort=oggetto javascript,
  description={sono delle strutture dati che possiamo definire all'interno del nostro codice.}
}
\newglossaryentry{array di oggetti}
{
  name=\glslink{array di oggetti}{array di oggetti},
  text=array di oggetti,
  sort=array di oggetti,
  description={sono delle collezioni non ordinate di oggetti javascript, sulle quali possiamo eseguire operazioni di vario tipo, aggiungere/rimuovere/modificare dati per esempio}
}

%%%%%%%%%%%%%%%%%%%%%%%%%%%%%%%%%%%%%%%%%%%%%%%%%%%%%%%%%
% \newglossaryentry{apig}
% {
%   name=\glslink{api}{API},
%   text=Application Program Interface,
%   sort=api,
%   description={in informatica con il termine \emph{Application Programming Interface API} (ing. interfaccia di programmazione di un'applicazione) si indica ogni insieme di procedure disponibili al programmatore, di solito raggruppate a formare un set di strumenti specifici per l'espletamento di un determinato compito all'interno di un certo programma. La finalità è ottenere un'astrazione, di solito tra l'hardware e il programmatore o tra software a basso e quello ad alto livello semplificando così il lavoro di programmazione}
% }

% \newglossaryentry{umlg}
% {
%   name=\glslink{uml}{UML},
%   text=UML,
%   sort=uml,
%   description={in ingegneria del software \emph{UML, Unified Modeling Language} (ing. linguaggio di modellazione unificato) è un linguaggio di modellazione e specifica basato sul paradigma object-oriented. L'\emph{UML} svolge un'importantissima funzione di ``lingua franca'' nella comunità della progettazione e programmazione a oggetti. Gran parte della letteratura di settore usa tale linguaggio per descrivere soluzioni analitiche e progettuali in modo sintetico e comprensibile a un vasto pubblico}
% }

