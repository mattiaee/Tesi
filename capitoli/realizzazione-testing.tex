% !TEX encoding = UTF-8
% !TEX TS-program = pdflatex
% !TEX root = ../tesi.tex

%**************************************************************
\chapter{Realizzazione e testing}
\label{cap:progettazione-codifica}
%**************************************************************

\intro{Breve introduzione al capitolo}\\

%**************************************************************

\section{Codifica}
Come ho agito in fase di codifica, cioè creazione dello store redux, creazione dell'api rtk, creazione dei vari componenti in react, integrazione dei componenti react, interazione tra componenti e stato/store test???
eslint-airbnb style guide utilizzate come standard aziendale
versionamento in gitlab regola aziendale 1 modifica 1 file 1 commit (nella situazione ideale), new file and delete file tutti sullo stesso commit purchè tutti dello stesso tipo
big commit iniziali o di modifiche più sostanziose 1 file 1 commit

Generalmente il flusso della codifica viene gestito in base allo sprint scrum, quindi lunedì retrospettiva su attività fatte, selezione delle attività da fare per lo scritp settimanale e si parte.
il ticket passa dallo stato "da completare" allo stato "in corso"
parlando di react che induce all'uso di componenti riutilizzabili, si è cercato di sfruttare il più possibile questo principio.
creazione di un nuovo componente
integrazione del componente nella vista / nelle viste
creazione dello store redux
creazione delle api redux-toolkit

flusso di lavoro in gitlab, con branch flow.. merge requeste per quando si deve approvare qualche modifica, ticket che passa in stato "da verificare"
se necessita di altre modifiche torna "in corso" altrimenti va in "completato"

parallelismo tra flow di git usato e ticketing jira con metodo scrum
documentazione nei ticket e nelle merge request di tutto quello che è stato fatto