% !TEX encoding = UTF-8
% !TEX TS-program = pdflatex
% !TEX root = ../tesi.tex

%**************************************************************
\chapter{Progettazione}
\label{cap:progettazione}
%**************************************************************

\intro{Breve introduzione al capitolo}\\

In questo capitolo vengono spiegate in modo dettagliate le tecnologie utilizzate nella realizzazione del progetto, inoltre viene data una spiegazione generale di come è
stata progettata l'architettura del sistema e una più approfondita riguardante il frontend, che era oggetto del progetto di stage.

%**************************************************************
\section{Tecnologie}
\label{sec:tecnologie-strumenti}

Di seguito viene data una panoramica delle tecnologie utilizzate.

  [[[forse posso dividere tra codice, testing, sistema, utility]]]
\subsection*{Javascript}
JavaScript è un linguaggio di programmazione orientato agli oggetti e agli eventi, comunemente utilizzato nella programmazione Web lato client per la creazione applicazioni web. L'intera applicazione è stata scritta con questo linguaggio.

\subsection*{React}
React è una libreria Javascript utilizzata per implementare interfacce utente (UI) lato frontend. React si basa sul concetto di component, idealmente è una libreria che permette di costruire i propri component come fossere degli elementi HTML del DOM per poi poterli riusare nell'intera applicazione.

\subsection*{Redux \& Redux RTK}
Redux è un contenitore dello stato per le applicazione Javascript. Viene usato per la gestione centralizzata dello stato delle applicazioni sviluppate in React Javascript. In particolare con la sua libreria Redux-Toolkit, permette una gestione dello stato semplice ed efficente.

\subsection*{API rest}
Descrizione Tecnologia 2

\subsection*{JSON}
Markup usato per la pubblicazione dei dati dell’API.

\subsection*{MUI}
Material UI è una libreria React open-source che permette di implementare i Google's Material Design. Essa comprende una collezione di componenti React precostruiti che possono essere facilmente adattati e messi in uso nella UI dell'applicazione.

\subsection*{React Router}
React Router è la libreria standard per il routing in React. Questa libreria permette la navigazione tra le varie viste dell'applicazione , permette di gestire le URL, e mantenere la sincronizzazione tra URL e viste.

\subsection*{Docker}
Docker è un sistema che permette di facilitare il deployment delle applicazioni.

\subsection*{jest} (???)
Descrizione Tecnologia 2

\subsection*{React testing library} (???)
Descrizione Tecnologia 2

\subsection*{airbnb js guidelines} (???)
Descrizione Tecnologia 2

\subsection*{eslint} (???)

%**************************************************************
\section{Architettura dell'applicazione}
L'architettura generale dell'applicazione era già stata progettata prima dell'inizio dello stage in oggetto, e in parte già esistente. Il sistema si divide principalmente in tre parti:
\begin{itemize}
  \item \textbf{backend}:
  \item \textbf{frontend}:
  \item \textbf{object storage}:
\end{itemize}

La comunicazione tra le diverse parti del sistema invece avviene con l'utilizzo di \textbf{APIrest}.

[disegno generale di front end back end api minio ecc ecc]
\subsection*{Backend}
Il backend è un'applicazione a se stante realizzata interamente in Python, con architettura MVC (Model-View-Controller). Questa architettura permette di separare completamente la logica del prodotto
dal modello, le viste non sono state approfondite in quanto lo scopo principale del backend è gestire e memorizzarei dati che vengono passati dal frontend e non visualizzarli. Il backend espone delle
particolari URL come endpoint per le chiamate REST del client frontend.
Il backend utilizza il modello (Model) per rappresentare i dati di interesse, il controller per gestirli, e la vista (View) per rappresentarli. Nel nostro caso non dovendo rappresentare i dati, operazione che spetta al frontend,
esso espone dell URL come endpoint per le chiamate REST proprio del frontend. In questo modo ogni volta che viene fatta una rechiesta su una corretta URL al backend, esso la interpreta con l'apposito controller, gestendo i dati strutturati come nel modello, e dopo aver
completato la gestione della richiesta, ne restituisce la risposta al frontend.
\subsection*{Frontend}
Il frontend dell'applicazione è generalmente la parte di interfaccia per l'utente, ossia quello che l'utente visualizza del nostro sistema e che gli da l'opportunità di interagire con il backend. Nel nostro caso il frontend
non fa solo da interfaccia utente ma gestisce anche dei dati nel proprio stato fin tanto che questi rimangono in sessione, in modo da non fare continue richieste al backend che rallenterebbero molto il sistema a causa della mole dei dati da gestire e visto che non tutti i dati devono essere passati al backend.
Il frontend è stato scritto in JavaScript e in particolare usando il framework React, gestendo però lo stato esternamente usando Redux, questo porta ad un'architettura leggermente più complessa dell'intero frontend ma da grandi benefici in quanto aiuta a mantenere per quanto possibile la separazione tra logica dell'applicazione
e reppresentazione dei dati.
  [disegno frontend react + redux]
In generale l'architettura è quella della single page application, cioè una pagina che non necessita di essere totalmente ricaricata ad ogni modifica ma che va a modificare soltanto la porzione interessata.
\subsection*{Object storage}
L'object storage che si è deciso di utilizzare è minIO (scelta aziendale, legata anche ad altri progetti) e nel caso della nostra applicazione serve in particolare come stumento per memorizzare i file con delle semplici richieste tramite API.
Questo ci permette essenzialmente due cose molto importanti non dover preoccuparci troppo dei dettagli implementativi della memeorizzazione e allo stesso tempo poter
reperire agevolmente i file che ci servono. Per il caricamento di un file è necessario fare una semplice richiesta PUT, ad una apposita URL minIO passando il file come parametro il file che si desidera salvare.
Questa URL viene concordata tra backend e minIO stesso in modo da essere sempre univoca per ogni nuovo file caricato, questo comporta che l'operazione finale di salvataggio del file comprenda tre richieste:
\begin{itemize}
  \item \textbf{frontend -> backend}: con questa richiesta il frontend comunica quale file vuole salvare al backend che può così memorizzarne i dati per tenerne traccia qual'ora si voglia in futuro reperire;
  \item \textbf{backend -> minIO}: il backend richiede su un apposita URL a minIO di comunicargli la vera e propria URL univoca sulla quale fare la richiesta di salvataggio del file, dopo che l'ha ottenuta la da in risposta al frontend;
  \item \textbf{frontend -> minIO}: il frontend dopo aver ricevuto la corretta URL sulla quale effettuare la richiesta PUT può eseguirla passando il file come parametro
\end{itemize}
Per la versione finale del progetto non è escluso di rivedere questa parte che si prevede una delle più importanti ma allo stesso tempo difficile da eseguire in modo efficace ed efficiente, per lo stato che il prodotto deve avere alla fine dello stage invece
questa gestione è più che sufficiente.
\subsection*{APIrest}

Le API rest sono il tramite tra le due applicazioni, vengono gestite nella parte di frontend con redux-tolkit, e interrogano
il backend sugli appositi endpoint, per poi lasciare al frontend l'interpretazione del risultato. Sono state utilizzate chiamate di
tipo GET e POST. Inoltre una particolare chiamata di tipo PUT viene effettuata dal frontend per l'upload dei file su minIO.
%**************************************************************
\section{Architettura frontend}
Per capire nel dettaglio l'architettura del frontend ricorriamo al seguente schema di approfondimento.
  [disegno del frontend redux react come lavorano]
L'applicazione frontend ha un architettura semplice...


Il frontend dell'applicazione è stato progettato come una single page application, che mantiene seprarazione tra lo stato e la
user interface.
\subsection*{React}
\subsection*{Redux}
\subsection*{React router}
\subsection*{Mui}

%**************************************************************
\section{Design Pattern utilizzati}
 (??????) tipo observer template se react lo usa... c'è una forma di MVVM in react+redux... posso citare MVC del back end o non frega niente a nessuno?

