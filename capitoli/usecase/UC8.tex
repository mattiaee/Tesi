% !TEX encoding = UTF-8
% !TEX TS-program = pdflatex
% !TEX root = ../tesi.tex

\subsection{UC8 - Visulizzazione degli Interventi relativi al file}
\begin{itemize}
  \item \textbf{Identificativo}: UC8
  \item \textbf{Nome}: visulizzazione degli Interventi relativi al file
  \item \textbf{Descrizione grafica}:
\end{itemize}

\begin{figure}[h]
  \centering
  %  \includegraphics[scale=0.50]{images/UC7.png}
  \caption{Descrizione grafica caso d'uso UC7}
\end{figure}

\begin{itemize}
  \item \textbf{Attori}
        \begin{itemize}
          \item \textit{Primari}: utente autorizzato
        \end{itemize}
  \item \textbf{Precondizione}: l'utente si trova sulla pagina per il caricamento dati, che ha già effettuato correttamente.
  \item \textbf{Postcondizione}: l'utente ha visualizzato gli Interventi relativi al file desiderato.
  \item \textbf{Scenario principale}: l'utente premendo sull'apposito bottone visualizza gli Interventi relativi al file.
  \item \textbf{Scenario secondario}: l'utente gestisce gli Interventi potendo modificare o rimuovere quelli esistenti oppure aggiungerne di nuovi.
\end{itemize}

\subsubsection{UC8.1 - Aggiunta Intervento}
\begin{itemize}
  \item \textbf{Identificativo}: UC8.1
  \item \textbf{Nome}: aggiunta Intervento
  \item \textbf{Descrizione grafica}: (approfondita in UC8)
  \item \textbf{Attori}
        \begin{itemize}
          \item \textit{Primari}: utente autorizzato
        \end{itemize}
  \item \textbf{Precondizione}: l'utente ha premuto il bottone per l'aggiunta di un Intervento.
  \item \textbf{Postcondizione}: l'utente visualizza il nuovo Intervento nell'elenco degli Interventi relativi al file.
  \item \textbf{Scenario principale}: l'utente compila il form per l'aggiunta del nuovo Intervento e lo salva.
\end{itemize}

\subsubsection{UC8.2 - Modifica Intervento}
\begin{itemize}
  \item \textbf{Identificativo}: UC7.2
  \item \textbf{Nome}: modifica Intervento
  \item \textbf{Descrizione grafica}: (approfondita in UC8)
  \item \textbf{Attori}
        \begin{itemize}
          \item \textit{Primari}: utente autorizzato
        \end{itemize}
  \item \textbf{Precondizione}: l'utente ha premuto il bottone per la modifica di un Intervento dall'elenco degli stessi.
  \item \textbf{Postcondizione}: l'utente visualizza il Intervento modificato nell'elenco degli Interventi relativi al file.
  \item \textbf{Scenario principale}: l'utente modifica i valori desiderati tramite il form per la modifica del Intervento e lo salva.
\end{itemize}

\subsubsection{UC8.3 - Eliminazione Intervento}
\begin{itemize}
  \item \textbf{Identificativo}: UC8.3
  \item \textbf{Nome}: eliminazione Intervento
  \item \textbf{Descrizione grafica}: (approfondita in UC8)
  \item \textbf{Attori}
        \begin{itemize}
          \item \textit{Primari}: utente autorizzato
        \end{itemize}
  \item \textbf{Precondizione}: l'utente ha premuto il bottone per l'eliminazione di un Intervento dall'elenco degli stessi.
  \item \textbf{Postcondizione}: l'utente visualizza l'elenco degli Interventi relativi al file senza l'Intervento eliminato.
  \item \textbf{Scenario principale}: l'utente elimina il Intervento scelto premendo il bottone per l'eliminazione.
\end{itemize}