% !TEX encoding = UTF-8
% !TEX TS-program = pdflatex
% !TEX root = ../tesi.tex

%**************************************************************
\chapter{Conclusioni}
\label{cap:conclusioni}
%**************************************************************
\section{Consuntiovo finale}
\label{sec:consuntivo-finale}

\begin{center}
  \renewcommand{\arraystretch}{1.8} %aumento ampiezza righe
  \begin{tabular}{ |p{1cm}|p{2,5cm}|p{8,5cm}| }
    %    \caption{Tabella del tracciamento dei requisti dello stage}
    %    \label{tab:requisiti-stage}
    \hline
    \multicolumn{2}{|c|}{\textbf{Settimana}} & \textbf{Attività}                                                                                                  \\
    \hline
    1                                        & dal 17/10/2022 al 25/10/2022 & autenticazione mediante server remoto                                               \\
    \hline
    2                                        & dal 17/10/2022 al 25/10/2022 & lettura dati da CD                                                                  \\
    \hline
    3                                        & dal 17/10/2022 al 25/10/2022 & precompilazione di form con i dati caricati da CD                                   \\
    \hline
    4                                        & dal 17/10/2022 al 25/10/2022 & editing dei dati del form                                                           \\
    \hline
    5                                        & dal 17/10/2022 al 25/10/2022 & upload dei dati verso i sistemi esterni                                             \\
    \hline
    6                                        & dal 17/10/2022 al 25/10/2022 & test di unità esaustivi                                                             \\
    \hline
    7                                        & dal 17/10/2022 al 25/10/2022 & possibilità di ascoltare le registrazioni                                           \\
    \hline
    8                                        & dal 17/10/2022 al 25/10/2022 & possibilità di modificare i dati mediante interazioni evolute (per es. drag-n-drop) \\
    \hline
    9                                        & dal 17/10/2022 al 25/10/2022 & realizzazione di un'applicazione desktop con Electron                               \\
    \hline
    10                                       & dal 17/10/2022 al 25/10/2022 & compilazione multipiattaforma dell'applicazione desktop                             \\
    \hline
  \end{tabular}
\end{center}

% \subsection*{Orario}

% \subsection*{Sviluppo}

\section{Raggiungimento degli obiettivi}
\label{sec:raggiungimento-obiettivi}
\begin{center}
  \renewcommand{\arraystretch}{1.8} %aumento ampiezza righe
  \begin{tabular}{ |p{1cm}|p{9cm}|p{2cm}| }
    %    \caption{Tabella del tracciamento dei requisti dello stage}
    %    \label{tab:requisiti-stage}
    \hline
    \textbf{Codice} & \textbf{Descrizione}                                                                & \textbf{Stato}          \\
    \hline
    O01             & autenticazione mediante server remoto                                               & soddisfatto             \\
    \hline
    O02             & lettura dati da CD                                                                  & soddisfatto             \\
    \hline
    O03             & precompilazione di form con i dati caricati da CD                                   & soddisfatto             \\
    \hline
    O04             & editing dei dati del form                                                           & soddisfatto             \\
    \hline
    D01             & upload dei dati verso i sistemi esterni                                             & soddisfatto             \\
    \hline
    D02             & test di unità esaustivi                                                             & soddisfatto             \\
    \hline
    F01             & possibilità di ascoltare le registrazioni                                           & soddisfatto             \\
    \hline
    F02             & possibilità di modificare i dati mediante interazioni evolute (per es. drag-n-drop) & non\newline soddisfatto \\
    \hline
    F03             & realizzazione di un'applicazione desktop con Electron                               & non\newline soddisfatto \\
    \hline
    F04             & compilazione multipiattaforma dell'applicazione desktop                             & non\newline soddisfatto \\
    \hline
  \end{tabular}
\end{center}

\section{Prodotto ottenuto}
\label{sec:prodotto-ottenuto}
Alla fine dell'esperienza di stage il prodotto, come già accennato nella parte introduttiva della relazione, si discosta leggermente dal quello che sarà il prodotto finale da
mettere in produzione. Si è deciso per motivi di tempo e di comprensione dell'argomento di realizzare le funzionalità che erano oggetto dello stage in prima battuta e svolgere in seconda
fase tutto quello che riguarda l'integrazione tra le varie parti del frontend, in particolare la creazione del ticket che alla fine dell'esperienza risulta non ancora completata.
Il prodotto non è ancora in produzione, ma è in un buono stato di avanzamento, a giudizio dell'azienda ospitante lo stage, e guardando al futuro si prevede che con qualche altro mese di
lavoro, potrà passare alla fase di collaudo.

\section{Conoscenze acquisite}
\label{sec:conoscenze-acquisite}
Questa esperienza di stage ha avuto grande valore per quanto riguarda l'acquisizione delle conoscenze tecniche e tecnologiche, infatti grazie all'aiuto in partocolare del tutor ma anche
di altri sviluppatore del team dell'azienda ospitante lo stage, posso dire di aver raggiunto una buona conoscenza degli strumenti utilizzati ma anche a lato pratico delle tecnologie
impiegate per lo sviluppo e i test. Guardando nel particolare, per qunato riguarda l'organizzazione aziendale e la gestione del progetto, ho acquisito una conoscenza approfondita nell'
utilizzo del framework SCRUM. Inoltre sempre per quanto riguarda la gestione del codice, un altro personale grande passo in avanti è stato fatto nella comprensione approfondita
di git della maggioranza delle sue funzionalità, in modo specifico per quanto riguarda il workflow utilizzato dall'azienda, il feature branch. A lato sviluppo e testing ovviamente l'apprendimento
è risultato più difficile perchè le tecnologie utilizzate erano completamente nuove in particolare Redux ma anche React, solo per citare le principali, ma in una valutazione retrospettiva
posso dire che il migliramento, nella comprensione dei paradigmi di sviluppo, e delle buone norme di codifica per le tecnologie utilizzate, è stato molto soddisfacente.

\section{Valutazione finale prodotto}
\label{sec:valutazione-finale-prodotto}
Il prodotto finale come detto in precedenza non risulta ancora utilizzato, ma essendo un progetto reale che l'azienda porterà avanti per mettere in seguito in produzione, continuerà ad
essere sviluppato per raggiungere quelli che sono gli obiettivi che l'azienda ha posto nell'analisi iniziale. Alla fine dell'esperienza di stage da parte mia sicuramente resta una piccola parte
di insoddisfazione per non aver raggiunto anche gli obiettivi posti come facoltativi che l'azineda porterà a termine nel prossimo futuro. La principale problematica riscontrata alla fine
dell'esperienza penso di poter dire che è la non conoscenza delle tecnologie utilizzate per lo sviluppo, nonostante un iniziale periodo di studio e prova, prima di iniziare a sviluppare
il prodotto vero e proprio, quando sono arrivato alla fine dello stage la mia conoscenza teorica e pratica è molto migliorata e questo mi fa guardare alla prime fasi di codifica con
insoddisfazione perchè con le conoscenze ora acquisite sicuramente queste parti si potevano svolgere in modo migliore, ecco perchè prima di ultimare lo stage parlando con il tutor ho
proposto varie soluzioni per migliorare la codifica del prodotto, per renderlo più solido, consistente, efficiente, efficace e manutenibile. Tutto questo costerà tempo all'azienda ma
probabilmente sarà del tempo che in futuro verrà risparmiato per eventuali future o manutenzioni da fare. Un altro punto su cui si è discusso con il referente aziendale è quello delle
funzionalità del prodotto, perchè come è normale che sia, in corso di sviluppo le funzinalità inizialmente previste vengono riviste e rivalutate. Per il nostro prodotto tutto quello
che era stato pensato in fase di analisi è stato tenuto come funzionalità da preservare anche per il futuro del prodotto, ma qualche aggiunta è stata pensata per renderlo più veloce e
pratico da utilizzare per i fonici, visto che l'obittivo del prodotto è proprio quello di facilitare il loro lavoro.

\section{Valutazione finale stage}
\label{sec:valutazione-finale-stage}
L'eperienza di stage è stata assolutamente positiva, mi ha dato modo di vedere a lato pratico molte delle cose che avevo visto in modo teorico nel corso degli studi. Ho effettuato lo
stage completamente da remoto, in modalità smart-working, e anche da questo punto di vista sono rimasto molto soddisfatto perchè ho potuto apprezzare il potenziale di questa modalità di
lavoro e secondo la mia opinione spinge maggiormente lo studente in stage a rendersi autonomo dal punto di vista lavorativo, ne migliora le capacità organizzative e consente la flessibilità
rispetto agli orari. In un modo di lavoro dove si guarda più agli obiettivi da raggiungere che al tempo effettivo di lavoro si incorre nel rischio di stimare male le tempistiche per le
attività da svolgere ma allo stesso tempo si rende maggiormente responsabile il lavoratore, e nel lungo periodo si valutano in modo più oggettivo le competenze e le capacità, è una
scelta che continuerei a fare anche nel mio futuro lavorativo. Un grande aiuto l'ho ricevuto dal mio tutor e da altri sviluppatori che con grande pazienza e conoscenze degli
argomenti molto approfondite mi hanno guidato in questo stage, senza mai lasciarmi ad affrontare i problemi del progetto da solo ma rimanendo sempre a disposizione per qualsiasi mio dubbio,
e mi ritengo molto soddisfatto anche per questo dell'azienda scelta per questa esperienza. L'unica vera difficoltà che ho incontrato in questo stage è la poca conoscenza iniziale delle
tecnologie, in particolare di React e ma in parte anche di git, che mi hanno costretto ad un inizio un po lento fatto più che altro di studio e approfondimento delle conoscenze, questo ha
comportato un rallentamento di tutte le attività del progetto, andando un po ad influire sulle tempistiche e il prodotto finale.