% !TEX encoding = UTF-8
% !TEX TS-program = pdflatex
% !TEX root = ../tesi.tex

%**************************************************************
\chapter{Conclusioni}
\label{cap:conclusioni}
%**************************************************************
\section{Consuntiovo finale}
\label{sec:consuntivo-finale}
tabella cronologica dello stage
e valutazione rispetto al gantt iniziale
esempio subsection

\begin{center}
  \renewcommand{\arraystretch}{1.8} %aumento ampiezza righe
  \begin{tabular}{ |p{1cm}|p{2,5cm}|p{8,5cm}| }
    %    \caption{Tabella del tracciamento dei requisti dello stage}
    %    \label{tab:requisiti-stage}
    \hline
    \multicolumn{2}{|c|}{\textbf{Settimana}} & \textbf{Attività}                                                                                                  \\
    \hline
    1                                        & dal 17/10/2022 al 25/10/2022 & autenticazione mediante server remoto                                               \\
    \hline
    2                                        & dal 17/10/2022 al 25/10/2022 & lettura dati da CD                                                                  \\
    \hline
    3                                        & dal 17/10/2022 al 25/10/2022 & precompilazione di form con i dati caricati da CD                                   \\
    \hline
    4                                        & dal 17/10/2022 al 25/10/2022 & editing dei dati del form                                                           \\
    \hline
    5                                        & dal 17/10/2022 al 25/10/2022 & upload dei dati verso i sistemi esterni                                             \\
    \hline
    6                                        & dal 17/10/2022 al 25/10/2022 & test di unità esaustivi                                                             \\
    \hline
    7                                        & dal 17/10/2022 al 25/10/2022 & possibilità di ascoltare le registrazioni                                           \\
    \hline
    8                                        & dal 17/10/2022 al 25/10/2022 & possibilità di modificare i dati mediante interazioni evolute (per es. drag-n-drop) \\
    \hline
    9                                        & dal 17/10/2022 al 25/10/2022 & realizzazione di un'applicazione desktop con Electron                               \\
    \hline
    10                                       & dal 17/10/2022 al 25/10/2022 & compilazione multipiattaforma dell'applicazione desktop                             \\
    \hline
  \end{tabular}
\end{center}

\subsection*{Orario}

\subsection*{Sviluppo}

\section{Raggiungimento degli obiettivi}
\label{sec:raggiungimento-obiettivi}
tabella requisiti stage superati e non
\begin{center}
  \renewcommand{\arraystretch}{1.8} %aumento ampiezza righe
  \begin{tabular}{ |p{1cm}|p{9cm}|p{2cm}| }
    %    \caption{Tabella del tracciamento dei requisti dello stage}
    %    \label{tab:requisiti-stage}
    \hline
    \textbf{Codice} & \textbf{Descrizione}                                                                & \textbf{Stato}          \\
    \hline
    O01             & autenticazione mediante server remoto                                               & soddisfatto             \\
    \hline
    O02             & lettura dati da CD                                                                  & soddisfatto             \\
    \hline
    O03             & precompilazione di form con i dati caricati da CD                                   & soddisfatto             \\
    \hline
    O04             & editing dei dati del form                                                           & soddisfatto             \\
    \hline
    D01             & upload dei dati verso i sistemi esterni                                             & soddisfatto             \\
    \hline
    D02             & test di unità esaustivi                                                             & soddisfatto             \\
    \hline
    F01             & possibilità di ascoltare le registrazioni                                           & soddisfatto             \\
    \hline
    F02             & possibilità di modificare i dati mediante interazioni evolute (per es. drag-n-drop) & non\newline soddisfatto \\
    \hline
    F03             & realizzazione di un'applicazione desktop con Electron                               & non\newline soddisfatto \\
    \hline
    F04             & compilazione multipiattaforma dell'applicazione desktop                             & non\newline soddisfatto \\
    \hline
  \end{tabular}
\end{center}

\section{Conoscenze acquisite}
\label{sec:conoscenze-acquisite}
dove ho acquisito conoscenze s git, programmazione js, vari react ecc ecc, metodologie di lavoro

\section{Valutazione finale}
\label{sec:valutazione-finale}
cazzate riguardo quanto è utile lo stage, come mi sono trovato con azienda tutor ecc ecc, valutazione critica delle conoscenze acquisite delle tempistiche
