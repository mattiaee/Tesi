% !TEX encoding = UTF-8
% !TEX TS-program = pdflatex
% !TEX root = ../tesi.tex


%**************************************************************
\chapter{Introduzione}
\label{cap:introduzione}
%**************************************************************

% \noindent Esempio di utilizzo di un termine nel glossario \\
% \gls{api}. \\

% \noindent Esempio di citazione in linea \\
% \cite{site:agile-manifesto}. \\

% \noindent Esempio di citazione nel pie' di pagina \\
% citazione\footcite{womak:lean-thinking} \

%**************************************************************%
%**************************************************************%
%**************************************************************%
\section{L'azienda}
\begin{figure}[H]
  \centering
  \includegraphics[width=2cm]{immagini/logo-SAI.jpg}
  \caption{Loso Syntema Artificial Intelligence}
\end{figure}
Synthema Artificial Intelligence (S.AI) è una \gls{start-up}\glsfirstoccur innovativa che si occupa di ricerca, progettazione, sviluppo, commercializzazione e manutenzione di
prodotti e servizi innovativi ad alto valore tecnologico, basati su tecniche di Intelligenza Artificiale
(in particolare reti neurali profonde), per l'analisi integrata e la comprensione di dati multimodali da fonti eterogenee (linguaggio naturale sia scritto che
parlato, audio, immagini, video, dati generati da sensori) e per la gestione dei flussi di lavoro, sia nel settore pubblico che privato.

%**************************************************************%
%**************************************************************%
%**************************************************************%
\section{Introduzione al progetto}
Il progetto nasce dalla necessità dell'azienda di creare un sistema semi-automatico per la creazione di particolari ticket. Il sistema è rivolto ai lavoratori della aule di tribunale,
ovvero i fonici, che si occupano di registrare su dei supporti CD quello che avviene nell'aula di un tribunale. Dopo aver registrato la loro parte di giornata nel tribunale di competenza,
devono inserire tutti i dati relativi ai processi a cui hanno assistito, nel sistema di memorizzazione della loro azienda. Le registrazioni audio dei fonici si articolano in una traccia per ogni microfono presente in aula
più una traccia mixer che li riassume tutti. La registrazione di queste tracce audio viene impostata ad inizio giornata, facendo corrispondere ad ogni microfono una traccia, per poi non essere più modificata.
Oltre a questo ogni fonico, con il sistema di registrazione presente in aula, può prendere degli appunti in riferimento a quello che succede, magari per alcuni casi particolari che si possono verificare o per ricordarsi i procedimenti a cui ha assistito.
Tutti questi \gls{metadati}\glsfirstoccur vengono salvati sottoforma di file di testo, questo file riporta tutto quello che è successo durante la registrazione in ordine cronologico, qual'è il microfono attivo e quindi chi sta parlando, se ci sono appunti particolari e così via.
Il lavoro del fonico si conclude poi con il caricamento di tutte le informazioni raccolte in aula sull'apposito sistema aziendale, con l'obiettivo di tenere traccia di tutti i procedimenti che avvengono in aula.
Questi dati vengono creati come ticket, ogni ticket si compone di un procedimento, che riporta un proprio codice e deve contenere varie informazioni (es: chi è o chi sono i giudici, imputati, PM).
Questo processo di creazione e caricamento di ticket al momento è molto laborioso per i fonici che devono sostanzialmente fare tutto a mano, l'obiettivo di questo progetto e del relativo prodotto finale è quello di
automatizzare questo processo, in modo che il fonico solo caricando l'intero CD contenente le tracce audio e il file di testo che le riassume visualizzi i vari ticket dei procedimenti a cui ha assistito. Così facendo il nostro prodotto
garantirebbe una notevole diminuzione del tempo di lavoro, una minor percentuale di errori e lascerebbe all'incaricato solamente l'onere di ricontrollare i dati relativi ai ticket automaticamente creati prima di caricarli sul sistema.

%**************************************************************%
%**************************************************************%
%**************************************************************%
\section{Principali problematiche}
Per questo progetto sorgono diverse problematiche che elenco di seguito:
\begin{itemize}
  \item \textbf{conoscenza dell'argomento}: le tematiche riguardanti i processi, le udienze e i procedimenti, risultano abbastanza vaghi allo start del progetto;
  \item \textbf{varietà dei metadati}: i metadati generati dal sistema, che coinvolgono lo schema delle tracce e gli appunti del fonico, possono generare grandi varietà di casi e sottocasi;
  \item \textbf{grandezza dei dati}: la mole dei dati da trattare in termini di \gls{byte}\glsfirstoccur, può raggiungere ordini di grandezza molto elevati,
        basti pensare che ogni ticket può comprendere svariate registrazioni e ognuna di esse può avere una durata molto lunga;
\end{itemize}

%**************************************************************%
%**************************************************************%
%**************************************************************%
\section{Soluzione scelta}
Il progetto è stato analizzato attentamente per scegliere la soluzione migliore, che sia allo stesso tempo veloce nel caricamento dei dati e nella creazione dei ticket, ma
che non perda informazioni importanti, visto la sensibilità dei dati trattati. Per questo progetto è stato scelto di creare un'applicazione inizialmente web ma con la possibilità
di renderla desktop in futuro. L'applicazione sarà formata da un \gls{frontend}\glsfirstoccur che avrà la principale funzione di gestire le informazioni necessarie a creare un ticket,
e un \gls{backend}\glsfirstoccur che servirà a tenerne traccia.
In questo modo ogni volta che un CD di registrazioni viene caricato, non viene direttamente mandato al backend, ma rimane in gestione al frontend che ne elabora i dati e li gestisce seguendo
le preferenze dell'utente, prima di passare solo i dati necessari al backend. Per la memorizzazione dei file invece la strada pensata in fase di analisi è l'utilizzo di un \gls{object storage}\glsfirstoccur
esterno dove fare l'upload solo delle tracce audio necessarie. L'oggetto del progetto di stage è proprio la parte di frontend di questo sistema, pensato per agevolare il lavoro dei
collaboratori delle aule di tribunale. Riassumendo, le soluzioni scelte a fronte delle principali problematiche riscontrate sono:
\begin{itemize}
  \item \textbf{conoscenza dell'argomento}: abbondante fase di studio iniziale e documentazione sulle tematiche poco note;
  \item \textbf{varietà metadati}:
        \begin{itemize}
          \item \textbf{\gls{libreria}\glsfirstoccur apposita}: creazione di una libreria interna apposita, con la funzione di analizzare e interpretare i metadati, in modo da comprenderne la maggior parte;
          \item \textbf{editing metadati}: possibilità di editare i metadati dopo che sono stati interpretati dalla libreria e visualizzati nella nostra applicazione, per permettere ai fonici di
                modificarli in fase di creazione del ticket;
        \end{itemize}
  \item \textbf{grandezza dei dati}:
        \begin{itemize}
          \item \textbf{gestione nel frontend}: i dati caricati vengono gestiti direttamente nel frontend, per poi passare al backend solo quelli realmente necessari alla creazione dei ticket;
          \item \textbf{object storage}: upload delle sole tracce audio che sono necessarie direttamente su un apposito object storage senza dover passarle al backend;
        \end{itemize}
\end{itemize}

%**************************************************************%
%**************************************************************%
%**************************************************************%
\section{Descrizione del prodotto ottenuto}
Il prodotto finale ottenuto si discosta leggermente da quello inizialmente pensato, infatti è stata soltanto abbozzata la costruzione del ticket vero e proprio per motivi legati alle
tempistiche dello stage e per dare la precedenza agli obiettivi obbligatori e desiderabili dello stage stesso.
L'applicazione finale risulta un buon punto di partenza per costruire la restante parte del prodotto. Il prodotto ottenuto ha come inizio l'autenticazione dell'utente con
la maschera di login, in modo che l'accesso a tutte le operazioni sia consentito solo agli utenti autorizzati. L'applicazione nello stato finale presenta le seguenti funzionalità:
\begin{itemize}
  \item \textbf{caricamento dei dati}: l'utente può caricare i dati sull'applicazione scegliendo se caricare interi CD di registrazioni o singoli file, questa seconda funzionalità
        risulta molto utile per gestire manualmente eventuali mancanze.
  \item \textbf{visualizzazione dati}: l'utente ha a disposizione la visualizzazione dei dati caricati in due versioni. Uno è un completo elenco dei file con i relativi metadati associati,
        l'altro è un elenco dei procedimenti che il sistema ha rilevato interpretando i metadati.
  \item \textbf{compilazione automatica \gls{form}\glsfirstoccur}: i form con i codici dei procedimenti, risultano automaticamente compilati, dopo che la libreria ha interpretato i metadati in fase di caricamento delle registrazioni.
  \item \textbf{riproduzione audio dei file}: nella pagina di visualizzazione dei file l'utente ha la possibilità di riprodurre ogni traccia audio caricata.
  \item \textbf{visulizzazione metadati del file}: grazie all'interpretazione dei metadati, per ogni file caricato è possibile vedere tutte le informazioni che ha a disposizione,
        nello specifico, l'utente, oltre a trovare tutti i procedimenti rilevati dalla libreria, può visualizzare tutti gli interventi e le annotazioni che sono state rilevate.
  \item \textbf{caricamento procedimenti}: dopo la revisione dei dati caricati, i procedimenti ritenuti validi possono essere facilmente caricati sul backend, insieme all'upload dei relativi file sull'object storage.
  \item \textbf{riepilogo dati salvati}: l'utente può visulizzare la lista dei procedimenti dei quali è già stato fatto l'upload e andare nel dettaglio del singolo procedimento per visualizzare le informazioni complete.
\end{itemize}


%**************************************************************%
%**************************************************************%
%**************************************************************%
\section{Strumenti utilizzati}
L'azienda usa un'infrastruttura già collaudata della quale fanno parte sistemi di versionamento del codice, strumenti per l'organizzazione del lavoro e per la comunicazione all'interno del gruppo di lavoro.
Di seguito una panoramica dettagliata degli strumenti utilizzati per lavorare nel team di sviluppatori dell'azienda.

%**************************************************************%
\subsection*{Slack}
\begin{figure}[H]
  \centering
  \includegraphics[width=2cm]{immagini/logo-Slack.jpg}
  \caption{Logo Slack}
\end{figure}
E lo strumento che viene usato per la collaborazione aziendale, permette di vedere lo stato degli utenti (sviluppatori dell'azienda
nel nostro caso) se disponibili o assenti al momento. Questa applicazione da la possibilità di comunicare singolarmente con gli altri
partecipanti oppure in piccoli gruppi divisi per progetto. Permette in oltre la condivisione dei file, molto utile nelle
comunicazioni veloci.

%**************************************************************%
\subsection*{Jira}
\begin{figure}[H]
  \centering
  \includegraphics[width=2cm]{immagini/logo-jira.jpg}
  \caption{Logo jira}
\end{figure}
Jira\footcite{site:jira} è lo strumento che l'azienda usa per l'organizzazione del lavoro. Permette di creare vari progetti, e per ogni progetto consente la
creazione dei relativi \gls{task}\glsfirstoccur da svolgere. In base alla metodologia di lavoro che si sceglie di utilizzare lo strumento permette di adottarla in tutto.
Nel nostro caso la scelta aziendale è il metodo SCRUM e questo strumento consente la creazione degli sprint, è fornito di un backlog, ha
una bacheca personalizzata per ogni utente che consente di vedere lo stato di avanzamento dei task di interesse. Inoltre consente di creare
report sull'andamento degli sprint oppure su alcuni periodi, cosa molto utile per tracciare la linea guida nel miglioramento aziendale.

%**************************************************************%
\subsection*{Gitlab}
\begin{figure}[H]
  \centering
  \includegraphics[width=2cm]{immagini/logo-Gitlab.jpg}
  \caption{Logo Gitlab}
\end{figure}
Sistema di versionamento dei file di codice usato dall'azienda. Questo strumento si basa su Git\footcite{site:git}.
I file vengono condivisi tra tutti gli svilupparori grazie a questo strumento, divisi in \gls{repository}\glsfirstoccur, uno per ogni progetto sul quale l'azienda lavora.
La linea guida aziendale per l'utilizzo del sistema Gitlab\footcite{site:gitlab} si basa sul \gls{workflow}\glsfirstoccur denominato feature branching, questo sistema di lavoro
prevede di lasciare il ramo principale (solitamente chiamato main) sempre pulito e con una versione funzionante e testata del prodotto, mentre invece
quando si vuole lavorare su una nuova feature si crea un nuovo ramo partendo da quello principale e una volta che questa feature sarà pronta per essere
revisionata e integrata con quella principale si chiederà una merge request di questo feature branch sul branch principale.
Il mantenimento dei file di codice sul repository si basa sulla regola aziendale 1 modifica - 1 commit - 1 file, ovvero si predilige che ogni branch che lo sviluppatore
crea per implementare la relativa feature sia composto da commit che riguardano soltato un file con la relativa modifica.

%**************************************************************%
\subsection*{VS Code}
\begin{figure}[H]
  \centering
  \includegraphics[width=2cm]{immagini/logo-VSCode.jpg}
  \caption{Logo VS code}
\end{figure}
E l'\gls{IDE}\glsfirstoccur di lavoro utilizzato per sviluppare in locale. Questo strumento è completo di tutto quello che serve per lo sviluppo del prodotto.
Esso infatti consente di sviluppare nei linguaggi che interessano il progetto, aiutando con vari \gls{plugin}\glsfirstoccur per il riconoscimento del codice lo sviluppatore a lavorare più velocemente e in modo più intuitivo. Inoltre è competamente integrato con i sistemi di
versionamento, in particolare nel nostro caso con GitLab. Oltre a queste funzionalità consente di utilizzare plugin per la pulizia del
codice, che settati in maniera opportuna permettono di rimanere sempre fedeli alle regole aziendali automaticamente ad ogni salvataggio.

%**************************************************************%
\subsection*{Jitsi}
\begin{figure}[H]
  \centering
  \includegraphics[width=2cm]{immagini/logo-Jitsi.jpg}
  \caption{Logo Jitsi}
\end{figure}
E lo strumento che l'azienda utilizza per le riunioni in videocall settimanali ma anche per quelle individuali che dovessero essere necessarie
in qualsiasi momento.

%**************************************************************%
\subsection*{SCRUM}
\begin{figure}[H]
  \centering
  \includegraphics[width=2cm]{immagini/logo-SCRUM.jpg}
  \caption{Logo SCRUM}
\end{figure}
SCRUM\footcite{site:scrum} è il \gls{framework}\glsfirstoccur che l'azienda segue per il lavoro in gruppo sui vari progetti. Questo framework molto diffuso per lo sviluppo software in
team si basa su sprint di durata breve che mirano al completamento dei task assegnati, il ripetersi temporale di questi sprint porta l'avanzamento
del prodotto. Più precisamente uno sprint è un arco di tempo nel quale si lavora sul completamento di alcune attività che vengono concordate per aggiungere valore al prodotto.
Queste attività, solitamente dette task, vengono prese dal backlog, ovvero una lista di attività che si vogliono svolgere per portare avanti il prodotto, il backlog si aggiorna con
l'avanzamento del progetto ed è quindi sempre in evoluzione.

Nel nostro caso particolare quella che segue l'azienda è una versione personalizzata di questo framework, il cui funzionamento è spiegato di seguito.
\begin{itemize}
  \item Durata degli sprint: normalmente una settimana;
  \item Meeting del lunedì:
        \begin{itemize}
          \item ognuno da una rapida panoramica per tenere tutti aggiornati su quello che ha svolto nella settimana precendente;
          \item si discutono e assegnano i task che si trovavano nel backlog per lo sprint seguente;
          \item rapida retrospettiva per valutazioni critiche dello sprint appena terminato;
        \end{itemize}
  \item Stati dei task da svolgere:
        \begin{itemize}
          \item da completare: nel corrente sprint ma nessuno ci sta ancora lavorando;
          \item in corso: qualcuno sta lavorando a questo task;
          \item pronto per la revisione: il task è stato completato, si attende che l'incaricato revisioni il codice e faccia il merge nel ramo principale;
          \item completato: il task è stato revisionato e integrato nel ramo principale;
          \item bloccato: per qualche ragione il task è bloccato, si scrive il motivo per il quale non può essere proseguito e se è necessario parlarne con qualcuno;
          \item da posticipare: il task deve essere riposto nel backlog e ripianificato per un altro sprint;
        \end{itemize}
\end{itemize}

%**************************************************************%
%**************************************************************%
%**************************************************************%
\section{Organizzazione del testo}

\begin{description}
  \item[{\hyperref[cap:descrizione-stage]{Il secondo capitolo}}] approfondisce la descrizione dello stage come è stato pianificato, quali sono gli oiettivi ed altri dettagli

  \item[{\hyperref[cap:analisi-requisiti]{Il terzo capitolo}}] approfondisce l'analisi dei requisiti in modo tecnico descrive come è stata svolta la fase di analisi

  \item[{\hyperref[cap:progettazione]{Il quarto capitolo}}] approfondisce la progettazione del prodotto descrivendo tecnicamente come è stata pensata l'architettura

  \item[{\hyperref[cap:realizzazione]{Il quinto capitolo}}] approfondisce la realizzazione del prodotto con la descrizione della fase di codifica e di testing

  \item[{\hyperref[cap:conclusioni]{Nel sesto capitolo}}] descrive le conclusioni valutando criticamente l'esperienza di stage e il prodotto ottenuto
\end{description}

Riguardo la stesura del testo, relativamente al documento sono state adottate le seguenti convenzioni tipografiche:
\begin{itemize}
  \item gli acronimi, le abbreviazioni e i termini ambigui o di uso non comune menzionati vengono definiti nel glossario, situato alla fine del presente documento;
  \item per la prima occorrenza dei termini riportati nel glossario viene utilizzata la seguente nomenclatura: \emph{parola}\glsfirstoccur;
\end{itemize}
