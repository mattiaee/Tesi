% !TEX encoding = UTF-8
% !TEX TS-program = pdflatex
% !TEX root = ../tesi.tex

%**************************************************************
\chapter{Analisi dei requisiti}
\label{cap:analisi-requisiti}
%**************************************************************

%\intro{Breve introduzione al capitolo}\\

\section{Casi d'uso}

Per lo studio dei casi di utilizzo del prodotto sono stati creati dei diagrammi.
I diagrammi dei casi d'uso (in inglese \emph{Use Case Diagram}) sono diagrammi di tipo \gls{uml} dedicati alla descrizione delle funzioni o servizi offerti da un sistema, così come sono percepiti e utilizzati dagli attori che interagiscono col sistema stesso.
Essendo il progetto finalizzato alla creazione di una applicazione per l'automatizzazione di un processo, le interazioni da parte dell'utente devono essere ridotte allo stretto necessario. Per questo motivo i diagrammi d'uso risultano semplici e in numero ridotto.
\newpage

% !TEX encoding = UTF-8
% !TEX TS-program = pdflatex
% !TEX root = ../tesi.tex

\subsection{UC1 - Autenticazione}
\begin{itemize}
  \item \textbf{Identificativo}: UC1
  \item \textbf{Nome}: autenticazione
  \item \textbf{Descrizione grafica}:
\end{itemize}

\begin{figure}[h]
  \centering
  %  \includegraphics[scale=0.50]{images/UC1.png}
  \caption{Descrizione grafica caso d'uso UC1}
\end{figure}

\begin{itemize}
  \item \textbf{Attori}
        \begin{itemize}
          \item \textit{Primari}: utente non autorizzato
          \item \textit{Secondari}: Google o Faceebook (???)
        \end{itemize}
  \item \textbf{Precondizione}: l'utente non autenticato si trova sulla pagina di autenticazione.
  \item \textbf{Postcondizione}: l'utente è autenticato.
  \item \textbf{Scenario principale}: l'utente vuole effettuare il login all'applicazione.
  \item \textbf{Scenario secondario}: l'utente non riesce ad autenticarsi a causa di un errore nella procedura. (\textbf{UC1.3})
\end{itemize}

\subsubsection{UC1.1 - Inserimento username}
\begin{itemize}
  \item \textbf{Identificativo}: UC1.1
  \item \textbf{Nome}: inserimento username
  \item \textbf{Descrizione grafica}: (approfondita in UC1)
  \item \textbf{Attori}
        \begin{itemize}
          \item \textit{Primari}: utente non autorizzato
        \end{itemize}
  \item \textbf{Precondizione}: l'utente ha a disposizione una username
  \item \textbf{Postcondizione}: l'utente ha inserito la username.
  \item \textbf{Scenario principale}: l'utente inserisce la username nell'apposito campo di input.
  \item \textbf{Scenario secondario}: l'utente ha inserito una username non corretta che causa un errore. (\textbf{UC1.3})
\end{itemize}

\subsubsection{UC1.1.1 - Errore inserimento username}
\begin{itemize}
  \item \textbf{Identificativo}: UC1.1.1
  \item \textbf{Nome}: errore inserimento username
  \item \textbf{Descrizione grafica}: (approfondita in UC1)
  \item \textbf{Attori}
        \begin{itemize}
          \item \textit{Primari}: utente non autorizzato
        \end{itemize}
  \item \textbf{Precondizione}: la username inserita dall'utente non è correttta.
  \item \textbf{Postcondizione}: l'errore viene mostrato all'utente.
  \item \textbf{Scenario principale}: l'utente inserisce una username non corretta, il sistema segnala l'errore all'utente e mostra nuovamente la maschera di login.
\end{itemize}

\subsubsection{UC1.2 - Inserimento password}
\begin{itemize}
  \item \textbf{Identificativo}: UC1.1
  \item \textbf{Nome}: inserimento password
  \item \textbf{Descrizione grafica}: (approfondita in UC1)
  \item \textbf{Attori}
        \begin{itemize}
          \item \textit{Primari}: utente non autorizzato
        \end{itemize}
  \item \textbf{Precondizione}: l'utente ha a disposizione una password
  \item \textbf{Postcondizione}: l'utente ha inserito la password.
  \item \textbf{Scenario principale}: l'utente inserisce la password nell'apposito campo di input.
  \item \textbf{Scenario secondario}: l'utente ha inserito una password non corretta che causa un errore. (\textbf{UC1.3})
\end{itemize}

\subsubsection{UC1.2.1 - Errore inserimento password}
\begin{itemize}
  \item \textbf{Identificativo}: UC1.2.1
  \item \textbf{Nome}: errore inserimento password
  \item \textbf{Descrizione grafica}: (approfondita in UC1)
  \item \textbf{Attori}
        \begin{itemize}
          \item \textit{Primari}: utente non autorizzato
        \end{itemize}
  \item \textbf{Precondizione}: la password inserita dall'utente non è correttta.
  \item \textbf{Postcondizione}: l'errore viene mostrato all'utente.
  \item \textbf{Scenario principale}: l'utente inserisce una password non corretta, il sistema segnala l'errore all'utente e mostra nuovamente la maschera di login.
\end{itemize}

\subsubsection{UC1.3 - Errore autenticazione}
\begin{itemize}
  \item \textbf{Identificativo}: UC1.3
  \item \textbf{Nome}: errore autenticazione
  \item \textbf{Descrizione grafica}: (approfondita in UC1)
  \item \textbf{Attori}
        \begin{itemize}
          \item \textit{Primari}: utente non autorizzato
        \end{itemize}
  \item \textbf{Precondizione}: il sitema di autenticazione riceve la richiesta da parte dell'utente.
  \item \textbf{Postcondizione}: il sistema comunica all'utente l'errore avvenuto, viene riproposta la maschera di login.
  \item \textbf{Scenario principale}: la richiesta di autenticazione non viene gestita dal sistema che comunica l'errore avvenuto all'utente e mostra nuovamente la maschera di login per un nuovo tentativo.
\end{itemize}
% !TEX encoding = UTF-8
% !TEX TS-program = pdflatex
% !TEX root = ../tesi.tex

\subsection{UC2 - Lettura dati da CD}
\begin{itemize}
  \item \textbf{Identificativo}: UC2
  \item \textbf{Nome}: lettura dati da CD
  \item \textbf{Descrizione grafica}:
\end{itemize}

\begin{figure}[h]
  \centering
  %  \includegraphics[scale=0.50]{images/UC2.png}
  \caption{Descrizione grafica caso d'uso UC2}
\end{figure}

\begin{itemize}
  \item \textbf{Attori}
        \begin{itemize}
          \item \textit{Primari}: utente autorizzato
        \end{itemize}
  \item \textbf{Precondizione}: l'utente autorizzato si trova nella pagina per il caricamento dei dati.
  \item \textbf{Postcondizione}: l'utente visualizza i dati che ha caricato.
  \item \textbf{Scenario principale}: l'utente premendo sull'apposito bottone può caricare i file contenuti nel CD di interesse.
  \item \textbf{Scenario secondario}: il sistema riscontra un errore nella procedura di caricamento dei dati. (\textbf{UC2.1})
\end{itemize}

\subsubsection{UC2.1 - Errore caricamento dati da CD}
\begin{itemize}
  \item \textbf{Identificativo}: UC2.1
  \item \textbf{Nome}: errore caricamento dati da CD
  \item \textbf{Descrizione grafica}: (approfondita in UC2)
  \item \textbf{Attori}
        \begin{itemize}
          \item \textit{Primari}: utente autorizzato
        \end{itemize}
  \item \textbf{Precondizione}: l'utente ha tentato di caricare i dati.
  \item \textbf{Postcondizione}: l'errore viene mostrato all'utente.
  \item \textbf{Scenario principale}: il sistema non è riuscito a gestire la richiesta di caricamento dei dati da parte dell'utente.
\end{itemize}
% !TEX encoding = UTF-8
% !TEX TS-program = pdflatex
% !TEX root = ../tesi.tex

\subsection{UC3 - Visualizzazione dati ordinati per file}
\begin{itemize}
  \item \textbf{Identificativo}: UC3
  \item \textbf{Nome}: visualizzazione dati ordinati per file
  \item \textbf{Descrizione grafica}:
\end{itemize}

\begin{figure}[h]
  \centering
  %  \includegraphics[scale=0.50]{images/UC3.png}
  \caption{Descrizione grafica caso d'uso UC3}
\end{figure}

\begin{itemize}
  \item \textbf{Attori}
        \begin{itemize}
          \item \textit{Primari}: utente autorizzato
        \end{itemize}
  \item \textbf{Precondizione}: l'utente si trova sulla pagina per il caricamento dati, che ha già effettuato correttamente.
  \item \textbf{Postcondizione}: l'utente visualizza i dati caricati sull'applicazione, ordinati per nome dei file.
  \item \textbf{Scenario principale}: l'utente ha caricato correttamente i dati e questi vengono visualizzati ordinati per nome del file.
\end{itemize}
\newpage
% !TEX encoding = UTF-8
% !TEX TS-program = pdflatex
% !TEX root = ../tesi.tex

\begin{usecase}{4}{Visualizzazione Dati ordinati per Procedimenti}
  \usecaseactors{Utente autorizzato}
  \usecasepre{L'utente si trova all'interno dell'applicazione}
  \usecasedesc{Permette la visualizzazione dei dati caricati da CD}
  \usecasepost{L'utente può visualizzare i dati caricati da CD ordinati per procedimenti}
  \label{uc:visualizzazione-dati-procedimenti}
\end{usecase}


\subsection{UC4 - Visualizzazione dati ordinati per procedimenti}
\begin{itemize}
  \item \textbf{Identificativo}: UC4
  \item \textbf{Nome}: visualizzazione dati ordinati per procedimenti
  \item \textbf{Descrizione grafica}:
\end{itemize}

\begin{figure}[h]
  \centering
  %  \includegraphics[scale=0.50]{images/UC4.png}
  \caption{Descrizione grafica caso d'uso UC4}
\end{figure}

\begin{itemize}
  \item \textbf{Attori}
        \begin{itemize}
          \item \textit{Primari}: utente autorizzato
        \end{itemize}
  \item \textbf{Precondizione}: l'utente si trova sulla pagina per il caricamento dati, che ha già effettuato correttamente.
  \item \textbf{Postcondizione}: l'utente visualizza i dati caricati sull'applicazione, ordinati per procedimenti.
  \item \textbf{Scenario principale}: l'utente ha caricato correttamente i dati e questi vengono visualizzati ordinati per procedimenti.
\end{itemize}
\newpage
% !TEX encoding = UTF-8
% !TEX TS-program = pdflatex
% !TEX root = ../tesi.tex
\subsection{UC5 - Elaborazione metadati}
\begin{itemize}
  \item \textbf{Identificativo}: UC5
  \item \textbf{Nome}: elaborazione metadati
  \item \textbf{Descrizione grafica}:
\end{itemize}

\begin{figure}[H]
  \centering
  \includegraphics[width=\textwidth]{immagini/usecase/UC5.png}
  \caption{Descrizione grafica caso d'uso UC5}
\end{figure}

\begin{itemize}
  \item \textbf{Attori}
        \begin{itemize}
          \item \textit{Primari}: utente autorizzato
        \end{itemize}
  \item \textbf{Precondizione}: l'utente si trova sulla pagina di visualizzazione dei file con i relativi metadati.
  \item \textbf{Postcondizione}: l'utente ha visualizzato ed eventualmente elaborato i metadati desiderati.
  \item \textbf{Scenario principale}: l'utente può visualizzare i metadati relativi ad ogni file caricato.
  \item \textbf{Scenario secondario}: l'utente può gestire (modificare aggiungere eliminare) i procedimenti relativi ad ogni file. (\textbf{UC5.1})
  \item \textbf{Scenario secondario}: l'utente può gestire (modificare aggiungere eliminare) i marker relativi ad ogni file. (\textbf{UC5.3})
  \item \textbf{Scenario secondario}: l'utente può gestire (modificare aggiungere eliminare) gli interventi relativi ad ogni file. (\textbf{UC5.4})
\end{itemize}

% \subsubsection{UC5.1 - Visualizzazione metadati}
% \begin{itemize}
%   \item \textbf{Identificativo}: UC5.1
%   \item \textbf{Nome}: visualizzazione metadati
%   \item \textbf{Descrizione grafica}: (approfondita in UC5)
%   \item \textbf{Attori}
%         \begin{itemize}
%           \item \textit{Primari}: utente autorizzato
%         \end{itemize}
%   \item \textbf{Precondizione}: l'utente vuole visualizzare i metadati relativi ad un file.
%   \item \textbf{Postcondizione}: l'utente visualizza i metadati del file.
%   \item \textbf{Scenario principale}: l'utente visualizza i metadati relativi al file come Marker, Interventi e Procedimenti.
%   \item \textbf{Scenario secondario}: l'utente può elaborare i metadati del file. (\textbf{UC5.2}, \textbf{UC5.4}, \textbf{UC5.5})
% \end{itemize}

\subsubsection{UC5.1 - Gestione metadati Procedimenti}
\begin{itemize}
  \item \textbf{Identificativo}: UC5.1
  \item \textbf{Nome}: getione metadati Procedimenti
  \item \textbf{Descrizione grafica}: (approfondita in UC5)
  \item \textbf{Attori}
        \begin{itemize}
          \item \textit{Primari}: utente autorizzato
        \end{itemize}
  \item \textbf{Precondizione}: l'utente vuole elaborare procedimenti relativi ad un file.
  \item \textbf{Postcondizione}: l'utente ha elaborato i procedimenti.
  \item \textbf{Scenario principale}: l'utente può modificare, aggiungere o eliminare i procedimenti relativi ad un file.
\end{itemize}

\subsubsection{UC5.2 - Upload procedimento}
\begin{itemize}
  \item \textbf{Identificativo}: UC5.2
  \item \textbf{Nome}: upload procedimento
  \item \textbf{Descrizione grafica}: (approfondita in UC5)
  \item \textbf{Attori}
        \begin{itemize}
          \item \textit{Primari}: utente autorizzato
        \end{itemize}
  \item \textbf{Precondizione}: l'utente vuole fare l'upload dei metadati riguardanti un procedimento relativi ad un file.
  \item \textbf{Postcondizione}: l'utente ha fatto l'upload del procedimento desiderato.
  \item \textbf{Scenario principale}: l'utente può tramite un apposito bottone fare l'upload del file e i relativi metadati del procedimento che desidera.
  \item \textbf{Scenario secondario}: si è verificato un errore nella richiesta di upload del procedimento. (\textbf{UC5.2.1})
\end{itemize}

\subsubsection{UC5.2.1 - Errore upload procedimento}
\begin{itemize}
  \item \textbf{Identificativo}: UC5.2.1
  \item \textbf{Nome}: errore upload procedimento
  \item \textbf{Descrizione grafica}: (approfondita in UC5)
  \item \textbf{Attori}
        \begin{itemize}
          \item \textit{Primari}: utente autorizzato
        \end{itemize}
  \item \textbf{Precondizione}: il sistema non ha gestito correttamente la richiesta di upload procedimento.
  \item \textbf{Postcondizione}: l'errore viene visualizzato sull'applicazione.
  \item \textbf{Scenario principale}: la richiesta di upload procedimento non va a buon fine e l'errore viene mostrato all'utente.
\end{itemize}


\subsubsection{UC5.3 - Gestione metadati Marker}
\begin{itemize}
  \item \textbf{Identificativo}: UC5.3
  \item \textbf{Nome}: getione metadati Marker
  \item \textbf{Descrizione grafica}: (approfondita in UC5)
  \item \textbf{Attori}
        \begin{itemize}
          \item \textit{Primari}: utente autorizzato
        \end{itemize}
  \item \textbf{Precondizione}: l'utente vuole elaborare i marker relativi ad un file.
  \item \textbf{Postcondizione}: l'utente ha elaborato i marker.
  \item \textbf{Scenario principale}: l'utente può modificare, aggiungere o eliminare i marker relativi ad un file.
\end{itemize}

\subsubsection{UC5.4 - Gestione metadati Interventi}
\begin{itemize}
  \item \textbf{Identificativo}: UC5.4
  \item \textbf{Nome}: getione metadati Intervetni
  \item \textbf{Descrizione grafica}: (approfondita in UC5)
  \item \textbf{Attori}
        \begin{itemize}
          \item \textit{Primari}: utente autorizzato
        \end{itemize}
  \item \textbf{Precondizione}: l'utente vuole elaborare gli interventi relativi ad un file.
  \item \textbf{Postcondizione}: l'utente ha elaborato gli interventi.
  \item \textbf{Scenario principale}: l'utente può modificare, aggiungere o eliminare gli interventi relativi ad un file.
\end{itemize}
\newpage
% !TEX encoding = UTF-8
% !TEX TS-program = pdflatex
% !TEX root = ../tesi.tex

\subsection{UC6 - Visualizzazione jobs}
\begin{itemize}
  \item \textbf{Identificativo}: UC6
  \item \textbf{Nome}: visualizzazione jobs
  \item \textbf{Descrizione grafica}:
\end{itemize}

\begin{figure}[h]
  \centering
  %  \includegraphics[scale=0.50]{images/UC6.png}
  \caption{Descrizione grafica caso d'uso UC6}
\end{figure}

\begin{itemize}
  \item \textbf{Attori}
        \begin{itemize}
          \item \textit{Primari}: utente autorizzato
        \end{itemize}
  \item \textbf{Precondizione}: l'utente si trova all'interno dell'applicazione e vuole visualizzare la lista dei jobs.
  \item \textbf{Postcondizione}: l'utente visualizza la lista dei jobs.
  \item \textbf{Scenario principale}: l'utente visualizza la lista dei jobs.
  \item \textbf{Scenario secondario}: l'utente può visualizzare il dettaglio di un singolo job premendo sull'apposito link. (\textbf{UC6.1})
\end{itemize}

\subsubsection{UC6.1 - Visualizzazione dettaglio job}
\begin{itemize}
  \item \textbf{Identificativo}: UC6.1
  \item \textbf{Nome}: visualizzazione dettaglio job
  \item \textbf{Descrizione grafica}: (approfondita in UC6)
  \item \textbf{Attori}
        \begin{itemize}
          \item \textit{Primari}: utente autorizzato
        \end{itemize}
  \item \textbf{Precondizione}: l'utente ha premuto sull'apposito link.
  \item \textbf{Postcondizione}: il dettaglio del job viene visualizzato.
  \item \textbf{Scenario principale}: l'utente visualizza tutti i dettagli del singolo job.
\end{itemize}

\section{Tracciamento dei requisiti}

Da un'attenta analisi dei requisiti e degli use case effettuata sul progetto è stata stilata la tabella che traccia i requisiti in rapporto agli use case.\\
Sono stati individuati diversi tipi di requisiti e si è quindi fatto utilizzo di un codice identificativo per distinguerli.\\
Il codice dei requisiti è così strutturato R(F/Q/V)(O/D/F) dove:
\begin{enumerate}
  \item[R =] requisito
  \item[F =] funzionale
  \item[Q =] qualitativo
  \item[V =] di vincolo
  \item[O =] obbligatorio
  \item[D =] desiderabile
  \item[F =] facoltativo
\end{enumerate}
Nelle tabelle \ref{tab:requisiti-funzionali} e \ref{tab:requisiti-qualitativi} sono riassunti i requisiti e il loro tracciamento con gli use case delineati in fase di analisi.
\newpage
\renewcommand{\arraystretch}{1.8} %aumento ampiezza righe
\begin{table}%
  \begin{tabularx}{\textwidth}{|l|X|c|}
    \hline
    \textbf{Requisito} & \textbf{Descrizione}                           & \textbf{Use Case} \\
    \hline
    RFO-1              & autenticazione                                 & UC1               \\
    \hline
    RFO-2              & lettura dati da CD                             & UC2               \\
    \hline
    RFO-3              & visualizzazione dati ordinati per file         & UC3               \\
    \hline
    RFO-4              & visualizzazione dati ordinati per procedimenti & UC4               \\
    \hline
    RFO-5              & gestione dei Marker relativi al file           & UC3               \\
    \hline
    RFO-6              & gestione degli Interventi relativi al file     & UC3               \\
    \hline
    RFD-1              & upload procedimento                            & UC5               \\
    \hline
    RFF-2              & visualizzazione jobs                           & UC6               \\
    \hline
    RFF-1              & riproduzione audio file                        & UC3.1             \\
    \hline
  \end{tabularx}
  \\
  \label{tab:requisiti-funzionali}
  \caption{Tabella del tracciamento dei requisti funzionali}
\end{table}%
\renewcommand{\arraystretch}{1.8} %aumento ampiezza righe
\begin{table}%
  \begin{tabularx}{\textwidth}{|l|X|l|}
    \hline
    \textbf{Requisito} & \textbf{Descrizione}    & \textbf{Use Case} \\
    \hline
    RQD-1              & test di unità esaustivi & -                 \\
    \hline
  \end{tabularx}
  \\
  \label{tab:requisiti-qualitativi}
  \caption{Tabella del tracciamento dei requisiti qualitativi}
\end{table}%